%% start of file `template_en.tex'.
%% Copyright 2007 Xavier Danaux (xdanaux@gmail.com).
%
% This work may be distributed and/or modified under the
% conditions of the LaTeX Project Public License version 1.3c,
% available at http://www.latex-project.org/lppl/.


\documentclass[11pt,a4paper]{moderncv}

% moderncv themes
% Here Blue is Blue itself, but not the blue itself.
% Noted.
\moderncvtheme[blue]{casual}                 % optional argument are 'blue' (default), 'orange', 'red', 'green', 'grey' and 'roman' (for roman fonts, instead of sans serif fonts)
%\moderncvtheme[green]{classic}                % idem

% character encoding
%\usepackage[utf8]{inputenc}                   % replace by the encoding you are using

% adjust the page margins
\usepackage[scale=0.8]{geometry}
\recomputelengths                             % required when changes are made to page layout lengths

% personal data
\firstname{中}
\familyname{秋} %这里中英文姓名的顺序不一样,注意一下就是了,不去改宏了。
%\title{Resumé title (optional)}               % optional, remove the line if not wanted
\address{君自故乡来}{应知故乡事}    % optional, remove the line if not wanted
\mobile{+86-13920091003}                    % optional, remove the line if not wanted
\phone{京V -02009}                      % optional, remove the line if not wanted
%\fax{fax (optional)}                          % optional, remove the line if not wanted
\email{lilybbs@live.cn}                      % optional, remove the line if not wanted
%\extrainfo{additional information (optional)} % optional, remove the line if not wanted
\photo[64pt]{yb}                         % '64pt' is the height the picture must be resized to and 'picture' is the name of the picture file; optional, remove the line if not wanted
\quote{这份档案 带给找工作的你 一份中秋的快乐}                 % optional, remove the line if not wanted



%这几行不再使用,仅做调试用,因为使用这几段可能会有错误,需要花时间去处理,在平时使用中可直接将此节删除。
%Noted 3:39:09 10月3日 2009年 
%Not yet used, 2nd, oct.2009.
%\nopagenumbers{}                             % uncomment to suppress automatic page numbering for CVs longer than one page
%%Added for chinese support
%\usepackage{hyperref}%不能有unicode选项,否则bookmark会是乱码
%\usepackage[pdftitle={Jerry Fleming's Resume},pdftex,bookmarks,bookmarksopen=true,colorlinks]{hyperref}
%Not yet used, 2nd, oct.2009.


%------------------------------------------------------------------------------------------------------
%U no, it's happy to use chinese now.
%需要注明的是,这种方法是我暂时惟一发现中文XeLaTeX对moderncv原样支持的方式。
%另外几种方式,我得花时间再去试一下,有很多问题,应该只是宏包的冲突。
% 2009-10-3 3:41:57 Saturday
\usepackage{xltxtra,fontspec,xunicode}    %Xelatex支持
\usepackage{zhfont}                       %用zhspacing包支持,之后我再去看 另一种实现是否有问题。
\zhspacing


%----------------------------------------------------------------------------------------
% 以下四行是设置可能用到的字体,但若无其它设定,本文档所用之默认字体为
% $zhspacing/zhfont.sty文件中的第71行附近,关于zhsffont的定义
% \newfontfamilywithslant\zhsffont{Adobe Heiti Std R}
% 即以下四行配置无实际作用。zhfont中设定优先级更高。
% zhspacing文档的解释是对于XeLaTeX和XeLaTeX对应的宏方案不同。
% 亦不支持\font\myname={"SuSE"}型宏。
\setzhmainfont[BoldFont=Adobe Heiti Std R]{Adobe Song Std L}%主要的中文字体
\setmainfont{Arial}           %英文字体
\setmonofont{Nimbus Sans L}                     %英文等宽体
\setsansfont{Nimbus Sans L}    %不注释了 线性非线性,这边没有限制。
%----------------------------------------------------------------------------------------
% 当然,可以在文本档第14行处,将roman/san 选项换一下,就可以更换main字体了,
% 但这并不能改变全文字体单调的囧境。
% 2009-10-3 3:44:41 Saturday
%一个折衷的方案,另定义字体宏。
\usepackage{fontspec}  

\newfontfamily\ni{"SimHei"}




%这里可以任意定义名称,只要愿意。


\usepackage{fancyhdr}
\usepackage{color}

%正如所看到的,这里hyperref的定义如下,可以将两个hypersetup合并,但这里两个略有不同些。
%%%第二个setup默认无效,如果要使其生效,可以注释掉moderncv模板137行处相对应的设置。
%这几处设置使hyperref支持中文。
%2009-10-3 3:51:05 Saturday

\RequirePackage{hyperref}
\hypersetup{%
  bookmarksnumbered=true,
  CJKbookmarks=true,
  bookmarksopen=false,
  bookmarksopenlevel=1,
  breaklinks=true,
  colorlinks=false,
  plainpages=false,
  pdfpagelabels=true,
  pdfborder=0 0 0}

  %%要使一下段生效,将moderncv模板137行处相应行注释掉。
  %%如果你的系统已经安装moderncv模板,那么它的可能位置是
  %%$TEXHOME\texmf-dist\tex\latex\moderncv\moderncv.cls
  %%2009-10-3 3:50:52 Saturday
  
    \hypersetup{%
      pdfauthor     = SinoSuSE,
      pdftitle      = 你的简历,%
      pdfsubject    = 中秋佳节 人人欢喜,%
      pdfkeywords   = 相濡以沫 为 相亲相爱,
	  }  
	  
	  
%不再使用。 2009-10-3 3:53:28 Saturday	  
%这边,SUSE拿来了几个字体大小使用的定义,这些定义全部来自ThuThesis。
%不再使用,只为对比参照磅值。 
%10月3日 2009年
% \suse@define@fontsize{chuhao}{42bp}
% \suse@define@fontsize{xiaochu}{36bp}
% \suse@define@fontsize{yihao}{26bp}
% \suse@define@fontsize{xiaoyi}{24bp}
% \suse@define@fontsize{erhao}{22bp}
% \suse@define@fontsize{xiaoer}{18bp}
% \suse@define@fontsize{sanhao}{16bp}
% \suse@define@fontsize{xiaosan}{15bp}
% \suse@define@fontsize{sihao}{14bp}
% \suse@define@fontsize{banxiaosi}{13bp}
% \suse@define@fontsize{xiaosi}{12bp}
% \suse@define@fontsize{dawu}{11bp}
% \suse@define@fontsize{wuhao}{10.5bp}
% \suse@define@fontsize{xiaowu}{9bp}
% \suse@define@fontsize{liuhao}{7.5bp}
% \suse@define@fontsize{xiaoliu}{6.5bp}
% \suse@define@fontsize{qihao}{5.5bp}
% \suse@define@fontsize{bahao}{5bp}

% 这里 再提供几个可以使用的命令,用来调整字体大小。
% 不提倡使用这些命令,因为这些命令的使用和字体磅数的联系并不明确。
% 在实际使用比较后才会有所判断选择。
% 如果有兴趣,相关定义可以在$texmf-dict/source/latex/base/classes.dtx中得到。
% # {\Huge          SinoSuSE}
% # {\huge          SinoSuSE}
% # {\LARGE         SinoSuSE}
% # {\Large         SinoSuSE}
% # {\large         SinoSuSE}
% # {\normalsize    SinoSuSE}
% # {\small         SinoSuSE}
% # {\footnotesize  SinoSuSE}
% # {\scriptsize    SinoSuSE}
% # {\tiny          SinoSuSE}



%-------------%--------------------------------------------------------------------------
%-------
%            content
%----------------------------------------------------------------------------------
\begin{document}
\maketitle


%%在这边 文档开始的地方,介绍一下这个命令,用来给每一个枚举项添加前置标志符。
%%\renewcommand{\listitemsymbol}{-} % change the symbol for lists


\section{教育背景}
%\cventry{year--year}{Degree}{Institution}{City}{\textit{Grade}}{Description}  % arguments 3 to 6 are optional
\cventry{2005.8-2009.12}{高力士}{}{一所大学}{一家学院}{2005-2009 有点奖学金    \hfill   倒没有挂科}  % arguments 3 to 6 are optional
\cventry{2005.8-2009.12}{圣斗士}{}{另一所大学}{XYZ学院}{ \color{green} Any other descriptions?}  % arguments 3 to 6 are optional


% 原来这一段,是可以套用来写发过的文章的。
% 就不删了,谁要是想用就直接用着吧。
% 10月3日
%\section{Thesis}
%\cvline{title}{\emph{Title}}
%\cvline{supervisors}{Supervisors}
%\cvline{description}{\small Short thesis abstract}

\section{项目经历}
%\subsection{Vocational}
%这是 cventry的一个例子,其实和上面 教育背景 栏中使用的cventry使用相同的范例
%\cventry{year--year}{Job title}{Employer}{City}{}{Description}                % arguments 3 to 6 are optional

\cventry{2009.5-2009.6}{月下独酌 }{李白}{三人}{}{吟得一首好诗。}

% 
%  这边举例可以用自己的字体插入并且调整字体大小,(具体字体的磅值和常见字号比可以参见本文档113行)。
%
%  方法1.直接插入所要字体,指定字体名称和大小
%  不再自定义宏实现。
% {\font\zhfont="SimSun" at 10bp  负责项目前期资料收集,撰写行业及竞争对手分析,筛选客户。主要客户为筹备上市中的中小型企业,多为生物,医疗器械等高新技术产业。}                
%  当然,如果自定义英文输入并自定义大小,也可以的。  
% {\font\rm="Consolas" at 14pt \rm This is SinoSuSE.}
%  Tips:如何知道自己想输入的字体名,输入fc-list>suse.txt 再打开suse.txt查找就可以了。
%
% 方法2. 替换main字体,在使用完后替换回来  
% \\
% \newfontfamilywithslant\zhsffont{SimSun}
% 中文\textbf{测试}。\textit{中文\textbf{测试}。}
% \newfontfamilywithslant\zhsffont{FangSong_GB2312}
% 中文\textbf{测试}。\textit{中文\textbf{测试}。}
% \newfontfamilywithslant\zhsffont{Adobe Heiti Std R}
%
% 方法3. 设定宏,宏的方式观赏性比较好。
% 宏的定义在 本文档76行
% \newfontfamily\ming{"STZhongsong"}
% 宏的使用在本文档186行
% {\ming   负责行业及竞争}
% 但这一方式的问题又很显然,因为没有办法直接指定字体大小
% 这里折衷的方案是 使用本文档106行处的字体大小控制宏。
%



\section{课外活动}
%这一段仍然使用之前 的模板。

\cventry{2006.9-2007.6}{\ni 春天}{}{不知细叶谁裁出}{}
					  {\font\zhfont="SimSun" at 10bp 1979年,那是一个春天 \\    
			            那一个圆圈圈,他就画在这边}
% 
						
\cventry{2007.11-2008.2}{夏天}{}{}{城上朱旗夏令初}	
					  {\font\zhfont="文泉驛正黑" at 10bp 还记得 大明湖畔的 夏雨荷么\\
			            蒲草韧如丝,磐石无转移,一点点可以}
                     

\cventry{2008.2-2008.5 }{秋天 }{}{为赋新词强说愁}{}	
					  {\font\zhfont="AR PL UMing TW" at 10bp  每两个星期去敬老院为老人演奏钢琴\\
			            参与图书义卖活动}
 

\cventry{ 2006.8-2008.5  }{冬天  }{}{雪上空留马行处}{}	
					  {\font\zhfont="Adobe Kaiti Std R" at 10bp 忽如一夜春风来 \\
			            一树梨花压海棠}
 

						
		
\section{英语水平}
\cvlanguage{\color{red} time 1}{国家英语四级考试}{同下}
\cvlanguage{\color{red} time 2}{国家英语六级考试}{同上}


 \section{获得奖励}
 \cvline{2005-2009}{\small 没有}
 \cvline{2005.9-2009.10}{\small 同上一下子}
 \cvline{}{\small 相濡以沫为相亲相爱}



% \section{Computer skills}
% \cvcomputer{category 1}{XXX, YYY, ZZZ}{category 4}{XXX, YYY, ZZZ}
% \cvcomputer{category 2}{XXX, YYY, ZZZ}{category 5}{XXX, YYY, ZZZ}
% \cvcomputer{category 3}{XXX, YYY, ZZZ}{category 6}{XXX, YYY, ZZZ}

% \section{Interests}
% \cvline{hobby 1}{\small Description}
% \cvline{hobby 2}{\small Description}
% \cvline{hobby 3}{\small Description}

% \closesection{}                   % needed to renewcommands
% \renewcommand{\listitemsymbol}{-} % change the symbol for lists

% \section{Extra 1}
% \cvlistitem{Item 1}
% \cvlistitem{Item 2}
% \cvlistitem[+]{Item 3}            % optional other symbol

% \section{Extra 2}
% \cvlistdoubleitem[\Neutral]{Item 1}{Item 4}
% \cvlistdoubleitem[\Neutral]{Item 2}{Item 5}
% \cvlistdoubleitem[\Neutral]{Item 3}{}
%  
% % 这边我们用不到bibtex,所以不用关心这一段。
% % 当然,如果要写论文的时候,还是需要好好读一读bibtex的
% % Bibtex三要素:(SuSE总结的。。。)
% % .bib file
% % \cite
% % \bibliographystyle{}&\bibliography{}
% % Publications from a BibTeX file
% \nocite{*}
% \bibliographystyle{plain}
% \bibliography{publications}       % 'publications' is the name of a BibTeX file

\end{document}

%% Tips:
%% 1. 平时使用换行符,为 \\。 即两个反候斜杠,但这个宏包下使用可能会有警告,虽然并不妨碍编译,但可以将换行符 换成 \newline{} 

%% end of file `template_en.tex'.
